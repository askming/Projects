\documentclass{article}
\usepackage[utf8]{inputenc}
\usepackage{fullpage}
\usepackage {setspace}
\usepackage[hang,flushmargin]{footmisc} %control footnote indent
\usepackage{url} % for website links
\usepackage{amssymb,amsmath}%for matrix
\usepackage{graphicx}%for figure
\usepackage{appendix}%for appendix
\usepackage{float}
\usepackage{multirow}
\usepackage{longtable}
\usepackage{morefloats}%in case there are too many float tables and figures
\usepackage{caption}
\usepackage{subcaption}
\usepackage{listings}
\captionsetup[subtable]{font=normal}
\usepackage{color}
\usepackage{hyperref}
\usepackage[round]{natbib}

%\usepackage{Sweave}
\setlength{\parindent}{0em}
\setlength{\parskip}{0.5em}


\graphicspath{{0.plots/}}



\begin{document}


\section{Background}
Longitudinal studies are ubiquitous in biostatistics context. For example, in a randomized clinical trials (RCT) where patients are randomly allocated into different treatment arms and are followed over time to collect some outcome of interest. Repeated measurements will then be produced from this follow-up mechanism. One of the important features of longitudinal data is that the repeated measurements from the same subject are more ``similar'' to each other compared to those measures from different subjects, i.e. within subject measures tend to be intercorrelated. This feature requires special statistical techniques to handle the correlation thus valid scientific inference can be drawn from the data. As discussed in \cite{diggle2002analysis}, there are mainly three methods we can use to analyze longitudinal data: marginal model, transition model and random effects model. Estimates of the regression coefficients from different models have different interpretations and the choice of a model depends on study objectives, the source of correlation as well as model's capacity. In this thesis work we will focus on applying random effects model to longitudinal data. A model that contains both random effects and fixed effects is called mixed effects model. The mixed effects model methodology, first introduced by R.A. Fisher \citep{fisher1919xv},  is a statistical tool that is used across a wide variety of disciplines including biostatistical contexts. Mixed effects models are especially popular in researches involving repeated measurements or observations from multilevel (or hierarchical) structure where the correlation between observations is not negligible as discussed above.\par

In many clinical trails and medical studies, time-to-event (survival) data is commonly generated along with the longitudinal measurements. Often times, the outcome of interest in survival data, for example disease recurrence, possible drop-outs, or death, is correlated with the longitudinal measurements. Simply ignoring and the correlation and fitting two models separately will lead to lost of information and misleading results. Joint model (JM) method for longitudinal and survival data was first proposed by \cite{tsiatis1995modeling} and \cite{faucett1996simultaneously} to handle this issue and to obtain unbiased estimators. JM is well studied in recent years, for example, \cite{henderson2000joint}, \cite{wang2001jointly}, and \cite{xu2001joint}. In terms of Bayesian methodology, \cite{guo2004separate} developed a fully Bayesian method to fit the JM using MCMC methods and implemented them in WinBUGS software. See \cite{yu2004joint} for a good review of JM topic. Many extensions have also been developed for JM, including considering multiple longitudinal outcomes \citep{brown2005flexible,rizopoulos2011bayesian}, incorporating multiple failure times \citep{elashoff2008joint}, and so on. \par


In most of the JM related works, the longitudinal part is modeled using linear mixed model (LMM), which is a widely used application of mixed effects methods. In brief, a LMM assumes the expected value of the outcome is a linear function of a set of covariates and observations from the same subject share a same unobserved latent variable, i.e. random effect, to account for the correlation among those observations. When conditional random effects, observations from the same subject are treated as independent. In addition, it also assumes the unobserved random error follows normal distribution.\par

Our concern for the widely used LMM is that in many circumstance the normality assumption of the error term can not always be satisfied, even after trying various ways of transformation. A commonly encountered situation is when there exits outliers or skewness in the outcome, where LMM is not appropriate to use. In other cases, the conditional mean may not be the primary interest and researchers may also be interested the covariate effects on the lower/upper quantiles of the outcome. \emph{For example}. Instead of trying to fix the limitations of LMM, quantile regression is another method that we may consider since it can meet all above needs directly. There are several advantages of quantile regression over the ubiquitous mean regression (a.n.a. linear regression) model. To list a few, quantile regression provides a much more comprehensive and focused insight into the association between the variables by studying the conditional quantiles functions of the outcome,  which may not be observed by looking only at conditional mean of the outcome \citep{koenker2005quantile}. In quantile regression, the regression coefficients (${\bf \beta}$) are functions of the quantile ($\tau$), and their estimates vary according to different quantiles. Thus quartile regression provides a way to studying the heterogeneity of the outcome that is associated with the covariates \citep{koenker2005quantile}. Moreover, as mentioned above, quantile regression is more robust against outliers in the outcome compared with the mean regression, which is an immediate extension from the property of quantiles. \par

Quantile regression is becoming more and more popular in the statistical community. \cite{koenker1978regression} introduced a method in estimating the conditional quantiles. As an introductory material, Koenker \cite{koenker2001quantile} briefly covers the fundamentals of quantile regression, parameter estimation techniques, inference, asymptotic theory, etc., his book provides more comprehensive and deeper introduction on quantile regression related topics \cite{koenker2005quantile}.  \cite{yu2001bayesian} introduced the idea of Bayesian quantile regression by modeling the error term using asymmetric Laplace distribution (ALD) followed the idea proposed in \cite{koenker1978regression}. \cite{kozumi2011gibbs} developed a Gibbs sampling algorithm for Bayesian quantile regression models, in which they used a location-scale mixture representation of the ALD. Many works have been done to extend the quantile regression method to accommodate longitudinal data. \cite{jung1996quasi} developed a quansi-likehood method for median regression model for longitudinal data. \cite{geraci2007quantile} proposed to fit the quantile regression for longitudinal data based on ALD and the estimation is made by using Monte Carlo EM algorithm. Later on, \cite{liu2009mixed} followed the idea of \cite{geraci2007quantile} and extended the model from random intercept to including random slope as well. The study of longitudinal data using quantile regression is becoming popular in recent years. \cite{fu2012quantile} proposed  a working correlation model for quantile regression for longitudinal data, a induced smoothing method was used to make the inference of the estimators. Fully Bayesian techniques and  Gibbs sampling algorithm became possible when the error term is decomposed as the mixture of normal and exponential random variables for the quantile linear mixed model  \citep{kotz2001laplace}, see \cite{kozumi2011gibbs} and \cite{luo2012bayesian} for applications. The fully Bayesian method is appealing to us because it is easy to implement and to make inference, the uncertainty of the unknowns is taken into account, and it is flexible in the distribution of random effects. The detailed background about Bayesian quantile linear mixed model will be introduced in Section \ref{sec:BLQMM}.\par



The joint model of longitudinal data, using quantile regression, and time to event data is little studied so far. To our knowledge, \cite{Alessio2014qrjm} is the first work that extended classical JM to incorporate quantile regression model in the longitudinal process. In their paper the parameter estimations is obtained using Monte Carlo expectation and maximization (MCEM) method.  Also, in recent years, another extension that attracts a lot of attention of the JM method is to make subject specific prediction for longitudinal or survival outcome based on model estimations \citep{rizopoulos2011dynamic,taylor2013real}. In this thesis project, we will focus on developing new extension to current joint models in analyzing longitudinal and time to event data. Specifically, first of all we will develop a fully Bayesian method in estimating the model parameters in the proposed JM; second, based on the estimations made on step one and given the historical information, dynamic prediction of the probability of event occurrence will be made and our method for prediction will be validated; at last, there will be an application to the real data  using our newly developed methods. \par





% Another concern that frequently arises from longitudinal study is that the subjects may be lost at follow up due to events. Such cases are called informative drop-outs (ID) when the event occurrences are associated with the longitudinal outcome. Simply ignore the missing mechanism can lead to biased estimates of the parameters in the longitudinal model. The informative drop-out problem has attracted much attention from the statisticians and a wide array of methods have been proposed to handle this issue \cite{diggle1994informative} \cite{lipsitz1997quantile} \cite{touloumi2003comparison} \cite{yuan2010bayesian}. For more information in this research area, there are several good review papers of modeling longitudinal data with non-ignorable drop-outs, including \cite{diggle2007analysis} and \cite{hogan2004handling}. \emph{[To-do: put some review of previous work and their limitations and propose our model then state the advantage.]}


\par












\bibliographystyle{plainnat}%%%%%%%%%%%%%%%%%%%%
\addcontentsline{toc}{section}{References}
\bibliography{QRJM}
\end{document}