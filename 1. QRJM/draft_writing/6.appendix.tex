% \documentclass{article}
% \usepackage[utf8]{inputenc}
% \usepackage{fullpage}
% \usepackage {setspace}
% \usepackage[hang,flushmargin]{footmisc} %control footnote indent
% \usepackage{url} % for website links
% \usepackage{amssymb,amsmath}%for matrix
% \usepackage{graphicx}%for figure
% \usepackage{appendix}%for appendix
% \usepackage{float}
% \usepackage{multirow}
% \usepackage{longtable}
% \usepackage{morefloats}%in case there are too many float tables and figures
% \usepackage{caption}
% \usepackage{subcaption}
% \usepackage{listings}
% \captionsetup[subtable]{font=normal}
% \usepackage{color}
% \usepackage{hyperref}
% \usepackage[round]{natbib}
% \usepackage{appendix}
% \usepackage{listings}
% \usepackage{courier}
% \usepackage{color}

% %\definecolor{codegreen}{rgb}{0,0.6,0}
% \definecolor{codegray}{rgb}{0.5,0.5,0.5}
% \definecolor{codepurple}{rgb}{0.58,0,0.82}
% \definecolor{backcolour}{rgb}{0.95,0.95,0.92}
% \definecolor{codeblack}{rgb}{0,0,0}
 
% \lstdefinestyle{mystyle}{
%     %backgroundcolor=\color{backcolour},   
%     commentstyle=\color{codegray},
%     keywordstyle=\color{codeblack},
%     numberstyle=\tiny\color{codegray},
%     stringstyle=\color{codeblack},
%     basicstyle=\normalsize\ttfamily,
%     breakatwhitespace=false,         
%     breaklines=true,                 
%     captionpos=b,                    
%     keepspaces=true,                 
%     numbers=left,                    
%     numbersep=5pt,                  
%     showspaces=false,                
%     showstringspaces=false,
%     showtabs=false,                  
%     tabsize=2
% }
 
% \lstset{style=mystyle}


% %\usepackage{Sweave}
% \setlength{\parindent}{0em}
% \setlength{\parskip}{0.5em}


% \graphicspath{{0.plots/}}



% \begin{document}


\begin{appendices}
\section{Appendix: simulation code and model files}
\subsection{R code to simulate data}
\begin{lstlisting}[language=R]
library(LaplacesDemon)
library(MASS)

###############################################################
############ function to simulate survival time ###############
###############################################################
# survival function is given by: S(t)= exp(- exp(B) * (exp(A*t) - 1) ) / A)
sim_Ti = function(n=500, alpha, delta=c(1,1), gamma=c(1,1)){
	Time = numeric(n)
	S = runif(n) # survival probability
	H = matrix(rnorm(2*n), ncol=2)
	W = matrix(rnorm(2*n), ncol=2)
	# random effects
	U = mvrnorm(n, mu=c(0,0), Sigma=matrix(c(0.09, 0.09*0.16, 0.09*0.16, 0.09), nrow=2, byrow=T))
	attributes(U)[[2]]=NULL # remove 'dimnames' attribute

	# calcualate survival time
	if(alpha[1]==0 & alpha[2]==0) Time = - log(S) / exp(gamma %*% t(W))
	
	else{
		B = alpha[1] * delta[1] * H[,1] + alpha[2] * U[,1] + gamma %*% t(W)
		A = alpha[2] * U[,2] + alpha[1] * delta[2] * H[,2]
		Time = log(1-log(S)*A/exp(B))/A 
	}

	Ti_id = which(!is.na(Time))
	Time = Time[Ti_id][1:250] # true survival time: take the first 250 that are not NA
	Ci = rbeta(250, 4, 1)*2 # censoring time
	Ti = pmin(Time, Ci) # observed survival time: choose the smaller one
	event = as.numeric(Time == Ti) # 1 for event, 0 for censor
	U = U[Ti_id, ][1:250, ]
	H = H[Ti_id, ][1:250, ]
	W = W[Ti_id, ][1:250, ]
		
	list(Ti=Ti, event=event, H=H, U=U, W=W)	
}

###############################################################
###### function to simulate longitudinal data #################
###############################################################
sim_longitudinal_data = function(survival_data=surdata, n=250, time=c(0, 0.25, 0.5, 0.75, 1, 3), tau, sigma=1, beta=c(1,1), delta=c(1,1)){
	# survival_data - data simulated from survival model
	# n - # of subjects
	# time - time points of observations
	# tau - quantile
	# sigma - scale parameter for ALD
	time = time # at most # = length(time) observations per patient
	y = matrix(NA, nrow=n, ncol=length(time)) # wide format
	Ti = survival_data$Ti
	U = survival_data$U # random effects
	H = survival_data$H
	X = cbind(1, rnorm(n))
	count = sapply(Ti, function(x) sum(x > time)) # number of observations after drop-outs

	for (i in 1:n){
		for (j in 1:count[i]){
			location = beta %*% X[i, ] + delta %*% c(H[i,1], H[i,2]*time[j]) + U[i,] %*% c(1, time[j])
			y[i,j] = ralaplace(1, location, scale=sigma, kappa=tau)		
		}	
	}
	
	list(y = y, X = X, J=count)		
}

###############################################################
###### function to simulate multiple joint data sets ##########
###############################################################
sim_multiple_data = function(N, sur_fun=sim_Ti, longi_fun=sim_longitudinal_data, alpha, tau){
	# N - number of data sets to generate
	# sur_fun - function to simulate survival data
	# longi_fun - function to sumualte longitudinal data
	# alpha - association parameters for JM
	# tau - quantile

	outdata = vector(mode='list', N)
	for (i in 1:N){
		sur_data = sur_fun(alpha=alpha)
		longi_data = longi_fun(sur_data, tau=tau)
		outdata[[i]] = list(survival_data=sur_data, longitudinal_data=longi_data)		
	}
	outdata	
}

\end{lstlisting}





\subsection{JAGS model file}
\begin{lstlisting}[language=python]
model{
	zero[1] <- 0
	zero[2] <- 0
	k1 <- (1-2*qt)/(qt*(1-qt))
	k2 <- 2/(qt*(1-qt))

	for (i in 1:I){
	# prior for random effects
	u[i, 1:2] ~ dmnorm(zero[], precision[,])

	# longitudinal process, BQR mixed model using ALD representation
		for (j in 1:J[i]){
			er[i,j] ~ dexp(sigma)
			mu[i,j] <- u[i,1] + u[i,2]*t[j] + inprod(X[i,], beta[]) + delta[1]*H[i,1] + delta[2]*H[i,2]*t[j] + k1*er[i,j]
			prec[i,j] <- sigma/(k2*er[i,j])
			y[i,j] ~ dnorm(mu[i,j], prec[i,j])
		} #end of j loop		

	# survival process, baseline hazard is set to 1
	A[i] <- alpha2*u[i,2] + alpha1*delta[2]*H[i,2]
	B[i] <- alpha1*delta[1]*H[i,1] + alpha2*u[i,1] + inprod(gamma, W[i,])
	S[i] <- exp(- exp(B[i])*(pow(exp(A[i]), Ti[i])-1)/A[i])
	h[i] <- exp(inprod(gamma, W[i,]) + alpha1*(delta[1]*H[i,1] + delta[2]*H[i,2]*Ti[i]) + alpha2*(u[i,1] + u[i,2]*Ti[i]))
	L[i] <- pow(h[i], event[i])*S[i]/1.0E+08

	# zero trick
	phi[i] <- -log(L[i])
	zeros[i] ~ dpois(phi[i])

	}#end of i loop

	precision[1:2,1:2] <- inverse(Sigma[,])
	Sigma[1,1] <- 1
	Sigma[1,2] <- rho*sig1
	Sigma[2,1] <- Sigma[1,2]
	Sigma[2,2] <- sig1*sig1

	# priors for other parameters
	alpha1 ~ dnorm(0, 0.001)
	alpha2 ~ dnorm(0, 0.001)
	beta[1] ~ dnorm(0, 0.001)
	beta[2] ~ dnorm(0, 0.001)
	delta[1] ~ dnorm(0, 0.001)
	delta[2] ~ dnorm(0, 0.001)
	gamma[1] ~ dnorm(0, 0.001)
	gamma[2] ~ dnorm(0, 0.001)
	sigma ~ dgamma(0.001, 0.001)
	rho ~ dunif(-1, 1)
	sig1 ~  dgamma(0.01, 0.01)
}

\end{lstlisting}




\end{appendices}


%All is done in \LaTeX \cite{knuth1986texbook}.
%
%
% \bibliographystyle{plainnat}%%%%%%%%%%%%%%%%%%%%
% \addcontentsline{toc}{section}{References}
% \bibliography{QRJM}


% \end{document}