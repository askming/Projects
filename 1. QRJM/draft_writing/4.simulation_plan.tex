\documentclass{article}
\usepackage[utf8]{inputenc}
\usepackage{fullpage}
\usepackage {setspace}
\usepackage[hang,flushmargin]{footmisc} %control footnote indent
\usepackage{url} % for website links
\usepackage{amssymb,amsmath}%for matrix
\usepackage{graphicx}%for figure
\usepackage{appendix}%for appendix
\usepackage{float}
\usepackage{multirow}
\usepackage{longtable}
\usepackage{morefloats}%in case there are too many float tables and figures
\usepackage{caption}
\usepackage{subcaption}
\usepackage{listings}
\captionsetup[subtable]{font=normal}
\usepackage{color}
\usepackage{hyperref}

%\usepackage{Sweave}
\setlength{\parindent}{0em}
\setlength{\parskip}{0.5em}


\graphicspath{{0.plots/}}



\begin{document}

\section{Plan for simulation study}

\begin{equation}\label{eqn:joint}
\left\{
\begin{array}{l}
y_{it} = \boldsymbol{\beta}^{\top}{\boldsymbol X}_{it} + \boldsymbol{\delta}^{\top}{\boldsymbol H}_{it} + {\boldsymbol u}_i^{\top}{\boldsymbol Z}_{it} + \varepsilon_{it} =\overset{\sim}{\tau}_{it} + \varepsilon_{it}\\
h(T_i|\mathcal{T}_{iT_i}, {\boldsymbol W}_i;  \boldsymbol{\gamma}, \alpha_1, 
\alpha_2) = h_0(T_i)\exp(\boldsymbol{\gamma}^{\top}{\boldsymbol W}_i + \alpha_1\boldsymbol{\delta}^{\top}{\boldsymbol H}_{iT_i} + \alpha_2{\boldsymbol u}_i^{\top}{\boldsymbol Z}_{iT_i})
\end{array}
\right.
\end{equation}


\section{Simulation setting }
Following the simulation setting from \cite{farcomeni2014longitudinal}, we set\par

$n=250$\par
$(\alpha_1, \alpha_2)=\{(1, 0), (0, 1), (0, 0)\}, \hspace{2em}\tau=\{0.25, 0.5, 0.75\}$\par
$\boldsymbol{\beta, \delta, \gamma}=(1, 1)^{\top}, \hspace{2em} \sigma=1$\par
$\boldsymbol{u}_i\sim N\left(\begin{pmatrix} 0\\ 0\end{pmatrix}, \begin{pmatrix}0.3^2 & 0.16*0.3^2\\ 0.16*0.3^2 & 0.3^2\end{pmatrix} \right)$\par

${\boldsymbol Z}_{it}=(1, t)^{\top},\hspace{1em} {\boldsymbol H}_{it}=(h_{i1}, h_{i2}*t)^{\top},\hspace{1em} {\boldsymbol X}_{it}=(1, x_i)^{\top},$ with $h_{i1}, h_{i2}, x_i, W_{i1}$ and $W_{i2}$ generated from independent standard normals. \par

Fix $h_0(s)=1$ and obtain the survival distribution as 

\[S(t|\boldsymbol{u}_i, {\boldsymbol H}_{it}, {\boldsymbol W}_i)=\exp\left\{-\frac{e^{\alpha_1(\delta_1H_{i1}+\delta_2H_{i2t}) + \alpha_2(u_{i1}+u_{i2}t) + \boldsymbol{\gamma}^{\top}\boldsymbol{W}_i} - e^{\alpha_1\delta_1H_{i1}+\alpha_2u_{i1} + \boldsymbol{\gamma}^{\top}\boldsymbol{W}_i}}{\alpha_2u_{i2}+\alpha_1\delta_2h_i}\right\}\] when $\alpha_1\ne0$ or $\alpha_2\ne0$ and 

\[S(t|\boldsymbol{u}_i, {\boldsymbol H}_{it}, {\boldsymbol W}_i) =  \exp\{-te^{\boldsymbol{\gamma}^{\top}\boldsymbol{W}_i}\}\]
when $\alpha_1=\alpha_2=0.$ We can obtain event time $T_i$ by inverting above survival function after generating $n$ random variates from standard uniform distribution.\par

Let the censoring time $C_i/5$ be distributed according to $beta(4,1)$ to obtain a censoring proportion around 25\%. \par

Longitudinal outcomes before drop out are independently generated from and ALD for the $\tau-$th quantile, centered on 
\[\boldsymbol{\beta}^{\top}{\boldsymbol X}_{it} + \boldsymbol{\delta}^{\top}{\boldsymbol H}_{it} + {\boldsymbol u}_i^{\top}{\boldsymbol Z}_{it},\] and with dispersion parameter $\sigma$.




\section{Result}
%%%%%%%%%%%%% insert table %%%%%%%%%%%%%%
\begin{table}[H]
\centering
\begin{tabular}{l|r}
Item & Quantity \\\hline
Widgets & 42 \\
Gadgets & 13
\end{tabular}
\caption{\label{tab:widgets}An example table.}
\end{table}

























%%%%%%%%%%%%% insert figure %%%%%%%%%%%%%%
%\begin{figure}[H]
%\begin{center}
%\includegraphics[scale=0.2]{figure.jpg}
%\caption{An example figure.}
%\end{center}
%\end{figure}
%
%
%All is done in \LaTeX \cite{knuth1986texbook}.
%
%
\bibliographystyle{plain}%%%%%%%%%%%%%%%%%%%%
\addcontentsline{toc}{section}{References}
\bibliography{QRJM}


























\end{document}