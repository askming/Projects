% \documentclass{article}
% \usepackage[utf8]{inputenc}
% \usepackage{fullpage}
% \usepackage {setspace}
% \usepackage[hang,flushmargin]{footmisc} %control footnote indent
% \usepackage{url} % for website links
% \usepackage{amssymb,amsmath}%for matrix
% \usepackage{graphicx}%for figure
% \usepackage{appendix}%for appendix
% \usepackage{float}
% \usepackage{multirow}
% \usepackage{longtable}
% \usepackage{morefloats}%in case there are too many float tables and figures
% \usepackage{caption}
% \usepackage{subcaption}
% \usepackage{listings}
% \captionsetup[subtable]{font=normal}
% \usepackage{color}
% \usepackage{hyperref}
% \usepackage[round]{natbib}

% %\usepackage{Sweave}
% \setlength{\parindent}{0em}
% \setlength{\parskip}{0.5em}


% \graphicspath{{0.plots/}}


% \begin{document}
\section{Data Description}\label{sec:data}

Hypertension induced heart failure (HF) is a major cause of hospitalization for American seninors. There are over 5.8 million people in the US suffering with HF and more than 670000 new cases joining this group every year. Hypertension treatments that have been studied include thiazide-type diuretics and angiotensin converting enzyme inhibitors (ACEIs), both of which have been shown to be effective in reducing the incidence of HF by treating hypertension.

As new treatment agents were developed, ALLHAT compared the effects of different treatment methods on fatal coronary heart disease (CHD) or non-fatal myocardial infarction (MI) among high-risk hypertensive patients. Four treatment methods under comparison were thiazide-type diuretic (chlorthalidone), ACEI (lisinopril), calcium channel-blocker (CCB; amlodipine), and $\alpha$-blocker (doxazosin), in which the later three are the newer antihypertensive agents whose effects were not previously studied. Under the multi-center, randomized, double-blind, active-controlled design mechanism, in the ALLHAT study a total of 42448 participants were randomized from 625 sites in the United States, Canada, Puerto Rico, and the US Virgin Islands \citep{grimm2001baseline}. The study included a large cohort of African Americans (36\% of total study population) and a relatively large proportion of Hispanic patients (19\%). Other major baseline characteristics of the study cohort include almost equal proportion of each gender (46.8\% women), average age of 67 years (with 35\% aged $\ge$70 years), high proportion of patients with diabetes (36\%), 47\% of the cohort with existing cardiovascular disease and 22\% are smokers. More detailed baseline characteristics of participants in the ALLHAT study can be found in \cite{grimm2001baseline}.\par

The full-scale ALLHAT study was started in fall of 1994 and the participants were then followed actively until March 31, 2002. Subsequently, the participants were then followed through 2006 in a national extended follow-up \citep{piller2011long}. As a result, the average of total follow-up time is 8.9 years: 4.9 years (3.2 years in the doxazaosin v.s. chlorthalidone comparison part) active follow-up time plus 4 years of extended follow-up. The primary outcome was cardiovascular mortality and the secondary outcomes were mortality, stroke, CHD, HF, cardiovascular disease, and end-stage renal disease. \cite{cushman2012mortality} has presented the mortality and morbidity results during and after the trial. During the active follow-up stage, outcomes as well as other clinical information like blood pressure are all available, however, during the post-trail follow-up no information is available for medications or blood pressure. In this thesis work, we will be using this ALLHAT dataset as the base population to make model estimations, based on which the subject specific dynamic predictions will be conducted either for patients from this study cohort or patients outside of the study cohort but with similar baseline characteristics.

% \bibliographystyle{plainnat}%%%%%%%%%%%%%%%%%%%%
% \addcontentsline{toc}{section}{References}
% \bibliography{QRJM}



























% \end{document}